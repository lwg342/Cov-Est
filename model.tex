The goal is to estimate \(\Sigma =\variance(y)\) where \(y\) is a \(p \times 1\) random vector, say asset returns. We collect the \(T\) observations into a matrix \(Y:p\times T\). Sample covariance estimate \(\hat{\Sigma} = \frac{1}{T} (Y - \bar{Y}\mathbf{1})(Y - \bar{Y}\mathbf{1})'\) is problematic when \(p\) is not small relative to \(T\). Popular estimation strategies include factor model, shrinkage, thresholding, banding, tapering, etc. 

If in addition to the observaion of \(Y\), we have identified from textual analysis a network \(G\) among the firms, where \(G_{ij} = 1\) implies that it's ``likely'' that the returns of firm \(i,j\) are correlated. For example, \cite{hoberg2016TextBasedNetwork} identifies a product similarity network from financial reports that have better performance than industry block diagonal matrix. Linked firms are potentially subject to similar demand shock in that specific industry. 

The question is how we can extract the information contained in these analyses to help estimate the covaraince \(\Sigma\). 

This paper proposes the following \textit{adaptive thresholding with auxiliary information} method, suppose we observe \(Y_{t}\) for \(t = 1 ,\dots, T\):
\begin{enumerate}
    \item Estimate the sample covariance estimate \(\hat{\Sigma}\), and the sample correlation matrix \(\hat{R}\). 
    \item Apply the generalized thresholding function \(h(r_{ij},\tau_{ij})\) to the off-diagonal elements of \(\hat{R} = (\hat{r}_{ij})\), \cite{rothman2009GeneralizedThresholding}. The novelty is now we allow the threshold \(\tau_{ij}\) to vary across elements and to depend on the network information.  Specifications we have considered for the threshold \(\tau\) are 
    \begin{itemize}
        \item 
        The probit model
            \begin{equation*}
            \tau_{ij} = \tau\pqty{G_{ij}} = \Phi\pqty{a + b \abs{G_{ij}}}
            \end{equation*}
        \item simple linear model 
            \begin{equation*}
                \tau(G_{ij}) = a + bG_{ij}
            \end{equation*}
    \end{itemize}
    \item Estimate the unknown parameters in the \(\tau\) function by cross validation, as in \cite{bickel2008CovarianceRegularization}, \cite{cai2011AdaptiveThresholding}, where we randomly split the sample \(V\) times, for each \(v\), compute the new estimator \(\hat{\Sigma}^{1,v}_{G}\) with the first subsample, and sample covariance \(\hat{\Sigma}^{2,v}\) and the criterion is 
    \begin{equation*}
        L(a, b) = \frac{1}{V} \sum_{v}^{V} \norm{\hat{\Sigma}^{1,v}_{G} - \hat{\Sigma}^{2,v}}^{2}_{F}
    \end{equation*}
    we find \(a,b\) that minimise this criterion. 
    
    \item Then with this estimation of \(a,b\), we can estimate \(\Sigma\) on the test sample. 
\end{enumerate}

There are several advantages of this method:
\begin{enumerate}
    \item The main advantage is that the network has economic meaning, comparing to data-driven thresholding or shrinkage. Hopefully, with a network that contains information about the real linkage, the relationship identified from auxialiary network will be more robust over time than the relationship identified from return data alone. 
    \item We can include multiple network estimates in the \(\tau\), perhaps also characteristics of the company, so it's both flexible and extensible.
    \item It might be helpful to capture industry-level comovement that is relevant to portfolio management and perhaps will capture some of the ``weak factors''.
\end{enumerate}

There are some challanges and questions as well:
\begin{enumerate}
    \item Since we don't observe the true \(\Sigma\), we wonder if there are some good criteria for evaluating the performance of our \(\hat{\Sigma}\). So far I am following \cite{ledoit2004HoneyShrunk} and \cite{ledoit2017NonlinearShrinkage}. But unlike the linear and nonlinear shrinkage method, our estimator is not designed for optimal performance under the Frobenius norm. 
    \item We don't have a quality measure of the network matrix or a model for how the \(G_{ij}\) are generated. 
    \item In simulation studies, the optimization takes some time, but the result is not yet very robust(some times it outperforms all competitors, but sometimes the optimization will be very   off-target).
\end{enumerate}
The following section will give detailed implementation of simulation results and some preliminary empirical.  